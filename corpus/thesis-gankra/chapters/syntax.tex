\chapter{Rust Syntax Primer}
\label{ch:syntax}

Since this thesis is focused on Rust in particular, it will include a lot of
Rust source code. Although our code is slightly unidiomatic
in order to avoid overloading the reader with the finer details of Rust's syntax
and semantics, this section exists for readers who are completely unfamiliar
with Rust. A significant portion of Rust's syntax and semantics will be
explained as it's needed throughout the thesis, but the following doesn't
really fit in anywhere. If one is interested in learning Rust properly,
\href{https://doc.rust-lang.org/book/}{\emph{The Rust Programming Language}}
is the officially recommended book, as it is free and written by the Rust developers.
If you're familiar with Rust already, feel free to skip this chapter.



\section{Basic Syntax}

First and foremost, Rust has the same basic control constructs and operators
as every other C-style language. \emph{if}, \emph{else}, \emph{while}, and
\emph{return} behave exactly as expected. Assignment is performed with $=$,
comparison with $==$, and so on. The only notable exception is that Rust does
not provide the traditional C-style for-loop. Instead, Rust only provides the
simpler for-each construct:

\begin{minted}{rust}
for x in iter {
    // do stuff with x
}
\end{minted}

Variables in Rust are declared with a \emph{let statement}. Variables are
immutable by default, and can be made mutable with the \emph{mut}
keyword. The type of a variable is usually inferred, but can be explicitly
specified with a colon.

\begin{minted}{rust}
let x = true;       // x is an immutable boolean
let mut y = false;  // y is a mutable boolean
let z: i32 = 0;     // z is an immutable 32-bit integer
\end{minted}

Functions in Rust are written as follows:

\begin{minted}{rust}
// A function called `add` that takes two 32-bit integers
// and returns their sum (also a 32-bit integer).
fn add(x: i32, y: i32) -> i32 {
    return x + y;
}
\end{minted}

There are two major composite data types in Rust, \emph{structs} and \emph{enums}.
Structs are exactly like structs in C -- a series of named and typed fields that
are all stored inline. Fields can be accessed through the \emph{dot operator},
as in most C-like languages.

\begin{minted}{rust}
// A struct called Foo
struct Foo {
    x: u32,         // x is a u32
    y: bool,        // y is a boolean
}

// Get the x field of the given Foo
fn get_x(data: Foo) -> u32 {
    return data.x;
}
\end{minted}

Enums, on the other hand, are a bit different from C.
Like C enums, Rust enums are a simple way to represent a type whose value can be
one of a small set of \emph{variants}. However Rust enums are
\emph{tagged unions}, which means each variant can have arbitrary data associated
with it.

\begin{minted}{rust}
// An enum called Bar with three variants A, B, and C
// A contains a u32
// B contains a boolean
// C has no data associated with it
enum Bar {
    A(u32),
    B(bool),
    C,
}
\end{minted}

In order to access the data in an enum's variant, one must pattern match against
the enum to determine what state it is in. Matching and binding the data to a variable
is performed as a single step to ensure that one does not access the data as if it
were a different variant.

\begin{minted}{rust}
// An exhaustive match of all possible variants that the value
// `data` could have.
match data {
    A(count) => {
        return count;
    },
    B(is_one) => if is_one {
        return 1;
    } else {
        return 0;
    },
    C => {
        return -1;
    },
}

// A fallible match of only a single variant. The code inside
// an `if-let` is executed only if `data` matches the pattern.
if let A(count) = data {
    return count;
}
\end{minted}

Rust uses $\&$ for the address-of operator, as well as the pointer type
itself. $*$ is the dereference operator. The dot operator will automatically
dereference pointers as necessary.

\begin{minted}{rust}
let x: i32  = 0;      // x is an i32
let y: &i32 = &x;     // y is a reference to an i32
let z: i32  = *y;     // z is an i32 (copied from x)

struct Foo { data: i32 }
let x: Foo = Foo { data: 0 };
let y: &Foo = &x;
let z: i32 = y.data;  // automatic dereferencing with `.`

\end{minted}





\section{Generics and Implementations}

Functions can be associated with a particular type by using an \emph{impl}
block. This is just a convenience for grouping functionality, and does not
imbue the code with any particular semantic. However it is necessary to
implement \emph{traits} which are Rust's version of an interface.

\begin{minted}{rust}
struct Foo {
    x: i32,
    y: i32,
}

// Associate the following code with a Foo
impl Foo {
    // A static function, in this case a constructor.
    // `new` has no special meaning in Rust, it's just a convention.
    // Invoked as `Foo::new()`
    fn new() -> Foo {
        return Foo { x: 0, y: 1 };
    }

    // A member function that takes a Foo by-reference
    // invoked as `some_foo.get_x()`
    fn get_x(&self) -> i32 {
        return self.x;
    }

    // A member function that takes a Foo by-value
    // invoked as `some_foo.consume()`
    fn consume(self) {
        let id = self.get_x();
        delete_the_database(id);
    }
}

// An interface for things that can be converted to integers.
trait ToInt {
    fn to_int(&self) -> i32;
}

// Implementing that interface for Foo
impl ToInt for Foo {
    // Trait implementations are invoked just like
    // any other member function
    fn to_int(&self) -> i32 {
        return self.x;
    }
}
\end{minted}



impl, fn, struct, enum, and trait can all be generic over arbitrary types.
Generics in Rust are implemented in a similar manner to C++ templates, but
are type-checked before any usage like Java generics. In order to facilitate
this, generic arguments can be specified to implement particular traits.

\begin{minted}{rust}
// `Compare` is generic over some type T, allowing
// implementors to be compared against multiple things.
trait Compare<T> {
    fn is_larger_than(&self, other: &T) -> bool;
}

// The `Generic` struct can be made for any type.
struct Generic<T> {
    data: T,
}

struct Concrete {
    data: i32,
}

// Implementing the Compare trait concretely
impl Compare<Concrete> for Concrete {
    fn is_larger_than(&self, other: &Concrete) -> bool {
        return self.data > other.data;
    }
}

// Because Compare is generic, we can implement it multiple
// times, as long as the generic arguments are different.
impl Compare<i32> for Concrete {
    fn is_larger_than(&self, other: &i32) -> bool {
        return self.data > other;
    }
}

// Implementing the Compare trait generically. We declare that
// this implementation applies only if T itself implements Compare
// with a `where` clause.
impl<T> Compare<Generic<T>> for Generic<T>
    where T: Compare<T>
{
    fn is_larger_than(&self, other: &Generic<T>) -> bool {
        return self.data.is_greater_than(&other.data);
    }
}

// A function generic over any two types which can be compared.
fn compare<R, L: Compare<R>>(left: &L, right: &R) -> bool {
    return left.is_larger_than(right);
}
\end{minted}

Traits implementations may have specific types associated with them.
For instance, if we wanted to make Compare incredibly generic, we could
make the result of \mintinline{rust}{is_larger_than} be defined by the
implementor:

\begin{minted}{rust}
trait Compare<T> {
    // The type of Result must be given by the implementor
    type Result;

    // And must be used as the return type for is_larger_than
    fn is_larger_than(&self, other: &T) -> Self::Result;
}

impl<T> Compare<Generic<T>> for Generic<T>
    where T: Compare<T>
{
    // My result type is bool
    type Result = bool;

    fn is_larger_than(&self, other: &Generic<T>) -> bool {
        return self.data.is_greater_than(&other.data);
    }
}
\end{minted}

Associated types are essentially the same as generic trait arguments,
except that they cannot be declared independently. Because Compare is
Generic over T, a type can declare that it can be compared to different things.
Previously we used this to specify that a Concrete could be compared to another
Concrete, or an i32. However associated types don't let us do this. The associated
types must be uniquely determined by the generic arguments. If a trait has no
generic arguments, but does have associated types, that means it can only be
implemented once per type.





\section{Closures}

Closures are anonymous functions that can refer to local variables that are
defined outside of the function. Closures are declared with pipes, as follows:

\begin{minted}{rust}
fn main() {
    let mut x = 0;
    // A closure that takes an `increment`
    // and increases `x` by that amount.
    let mut func = |increment: i32| { x += increment; };

    func(1);    // x += 1;
    func(2);    // x += 2;
}
\end{minted}

How values should be captured is generally inferred based on usage, but Rust
will only do this for capturing things by-reference. To indicate that something
should be captured by value, the move keyword must be used:

\begin{minted}{rust}
fn main() {
    let x = 0;

    // A closure that takes an `increment`,
    // increases `x` by that amount, and returns it.
    // Effectively takes a snapshot of what `x` was
    // when the closure was made.
    let func = move |increment: i32| {
        return x + increment;
    };

    let y = func(1);
    let z = func(2);

    // prints 1
    println!("{}", y);
    // prints 2, because each call uses a fresh copy of `x`
    println!("{}", z);
}
\end{minted}



