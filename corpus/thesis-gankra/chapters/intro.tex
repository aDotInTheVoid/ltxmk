\chapter{Introduction}
\label{ch:intro}

Modern systems are built upon shaky foundations. Core pieces of computing
infrastructure like kernels, system utilities, web browsers, and game engines
are almost invariably written in C and C++. Although these languages provide
their users the control necessary to satisfy their correctness and performance
requirements, this control comes at the cost of an overwhelming burden for the
user: \emph{Undefined Behaviour}. Users of these languages must be ever-vigilant
of Undefined Behaviour, and even well-maintained projects fail to do so
\cite{wang2012undefined}. The price of invoking Undefined Behaviour is dire:
compilers are allowed to do \emph{absolutely anything at all}.

Modern compilers can and will analyze programs on the assumption that Undefined
Behaviour is impossible, leading to counter-intuitive results like time travel, \cite{timetravel}
wherein correct code is removed because subsequent code is Undefined. This
aggressive misoptimization can turn relatively innocuous bugs like integer
overflow in a print statement into wildly incorrect behaviour. Severe
vulnerabilities subsequently occur because Undefined Behaviour often leads to
the program blindly stomping through memory and bypassing checks, providing attackers
great flexibility.

As a result, our core infrastructure is constantly broken and exploited. Worse,
C and C++ are holding back systems from being as efficient as they could be.
These languages were fundamentally designed for single-threaded programming and
reasoning, having only had concurrency grafted onto their semantics in the last
5 years. As such, legacy code bases struggle to take advantage of hardware
parallelism. Even in a single-threaded context,
the danger of Undefined Behaviour encourages programs to be written
inefficiently to defend against common errors. For instance it's much safer
to copy a buffer than have several disjoint locations share pointers to it,
because unmanaged pointers in C and C++ are wildly unsafe.

In order to address these problems, Mozilla developed the Rust programming language.
Rust has ambitious goals. It intends to be more efficient than C++ and safer
than Java without sacrificing ergonomics. This is possible because, as we discussed,
C++ code is often designed inefficiently in order to defend against its own unsafety.
Although Rust 1.0 was released less than a year ago \cite{rust1}, early results
are promising. To see this, we need look no further than the Servo project, a
rewrite of Firefox's core engine in Rust
\cite{servo}.

Preliminary results \cite{servo-exp} have found that Servo can
perform layout two times faster than Firefox on a single thread, and four times
faster with four threads. Servo has also attracted over 300 contributors
\cite{servo-gh}, in spite of the fact that it's not production ready and
written in a language that was barely just stabilized. This demonstrates that
Rust has \emph{some} amount of usability, and that applications written in it
can actually be maintained by novices (since almost everyone using Rust
is new to it). There's little data on safety and correctness in Servo, but we'll
see throughout this thesis that Rust provides powerful tools for ensuring
safety and correctness.

It should be noted that Rust doesn't desire to invent novel systems or analyses.
Research is hard, slow, and risky, and the academic community has already produced
a wealth of rigorously tested ideas that have little main-stream exposure. As a
result, to a well-versed programming language theorist Rust can be easily
summarized as a list of thoroughly studied ideas. However, Rust's type system
as a whole is greater than the sum of the parts. As such, we believe it merits
research of its own.

Truly understanding Rust's type system in a rigorous way is a massive
undertaking that will fill several PhD theses, and Derek Dreyer's lab is already
working hard on that problem \cite{rustbelt}. As such, we will make no effort
to do this. Instead, we will focus in on what we consider the most interesting
aspect of Rust: \emph{ownership}. Ownership is an emergent system built on three
major pieces: \emph{affine types}, \emph{region analysis}, and \emph{privacy}. Together we get
a system for programmers to manage where and when data lives, and where
and when it can be mutated, while still being reasonably ergonomic and intuitive
(although with an initially steep learning curve).

Ownership particularly shines when one wants to develop concurrent programs,
an infamously difficult problem. Using Rust's tools for generic programming,
it's possible to build totally safe concurrent algorithms and data structures
which work with arbitrary data without relying on any particular paradigm like
message-passing or persistence. Rust's crowning jewel in this area is the
ability to have child threads safely mutate data on a parent thread's stack
without any unnecessary synchronization, as the following program demonstrates.
(If you aren't familiar with Rust, a brief introduction is provided in the
next chapter)

\begin{minted}{rust}
[dependencies]
crossbeam = 0.1.6

-----

extern crate crossbeam;

fn main() {
    // An array of integers, stored on the stack
    let mut array = [1, 2, 3];

    // Create a scope to know when to join all threads
    crossbeam::scope(|scope| {

        // Get pointers to each element in the array
        for x in &mut array {
            // Spawn a thread to increment this element
            scope.spawn(move || {
                // Child thread mutates parent thread
                // This is neither atomic, nor specifically
                // synchronized.
                *x += 1;
            });
        }

        // Block on all the child threads joining here
    });

    println!("{:?}", array);
}
\end{minted}

The most amazing part of this program is that it's based entirely on a third-
party library (crossbeam \cite{crossbeam}). Rust does not require threads to be modeled as part
of the language or standard library in order for these powerful safe
abstractions to be constructed. All of the following operations won't just fail
to work, but will \emph{fail to compile}: sharing non-threadsafe data with
the scoped threads, giving the same pointer to two threads, keeping a pointer for too
long, or accessing the array while the threads are still running. The fact that
misuse is a static error is incredibly important for concurrent programs, because
runtime errors are more difficult debug due to the need to isolate and reproduce
the behaviour.

In order to understand how this program works and why it's safe, we'll need to
build up some foundations.

In chapter 2, we provide a brief introduction to some of Rust's syntax.

In chapter 3, we establish the basic principles of
safety and correctness that we're interested in. Since these are ultimately
vague concepts in a practical setting, we do this by surveying several classic
errors that all languages must deal with in one way or another.

In chapter 4 we provide an analysis of these problems in terms of \emph{trust}.
In particular, we observe that these problems can be reduced to how different
regions of code trust each other with data. We identify three major strategies for handling
these trust issues, and briefly categorize some common solutions to these problems
according to these strategies.

In chapter 5 we introduce ownership. We show how affine types, region
analysis, and privacy contribute to Rust's ability to model and enforce trust
between different pieces of code. Although this is mostly describing concepts
that are well-understood at an intuitive level in the Rust community, this
work is the first to actually write them down completely, and in terms of
trust.

In chapter 6 we analyze several standard Rust interfaces use ownership for trust.
We demonstrate how Rust's story surrounding indexing into arrays exactly aligns
with our strategies for trust. Of interest is that tagged unions use ownership
to provide values that must be checked for validity to be accessed, and iterators
use ownership to provide safe unchecked indexing. We use the \emph{entry} API,
which was developed as part of this work, to demonstrate how ownership can be
used to externally expose the execution of an algorithm, producing more flexible
designs that gives greater control to users.

In chapter 7 we discuss some of the limitations of Rust's version of ownership.
We show how ownership struggles with ad-hoc cyclic data structures, and some
of the community's solutions to this problem. We use the \emph{drain} API,
another interface developed as part of this work, to demonstrate how ownership
is mostly only able to model things that \emph{must not} be done by an interfaces
user, but struggles to model things that \emph{must} be done. Finally, we review
how the scoped thread API works.

In chapter 8 we briefly survey how ownership in Rust compares to similar systems
in other languages. In particular, Cyclone and C++, Rust's two closest relatives.
