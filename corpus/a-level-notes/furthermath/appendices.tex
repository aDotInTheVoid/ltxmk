\chapter{Formulae and Identities}
%TODO: Have header say appendix in appendix

These will not be given. You are expected to know them.

\section{Pure Mathematics}
\subsection{Quadratic Equations}
\[ax^2+bx+c=0 \iff  x=\frac{-b\pm\sqrt{b^2-4ac}}{2a}\]

\subsection{Laws of Indices} 
\begin{align*} 
    a^xa^y &\equiv a^{x+y}\\
    a^x \div a^y &\equiv a^{x-y}\\
(a^x)^y &\equiv a^{xy}
\end{align*}

\subsection{Laws of Logarithms}
\(
    x = a^n \iff n = \log_ax
\) where \(a>0\) and \(x>0\)
\begin{align*}
    \log_ax + \log_ay &\equiv \log_a(xy) \\
    \log_ax - \log_ay &\equiv \log_a(\frac{x}{y}) \\
    k\log_ax &\equiv \log_a(x^k)
\end{align*}

\subsection{Coordinate Geometry}

A straight line graph, gradient \(m\) passing through \( (x_1, y_1) \) has equation
\[
    y-y_1 = m(x-x_1)
\]
\noindent
Straight lines with gradients \(m_1\) and \(m_2\) are perpendicular when  \( m_1m_2=-1 \)

\subsection{Sequences}

General term of an arithmetic progression: 
\[u_n=a+(n-1)d\]
General term of a geometric progression:
\[u_n=ar^{n-1}\]

\subsection{Trigonometry}
In the triangle ABC:
\begin{align*}
    \frac{a}{\sin A} &= \frac{b}{\sin B} = \frac{c}{\sin C} \\
    a^2 &= b^2 + c^2 -2bc\cos A \\
    \text{Area} &= \frac{ab\sin C}{2}
\end{align*}

\begin{align*}
1 &\equiv \cos^2A+\sin^2A \\
\sec^2A&\equiv 1+\tan^2A\\
\csc^2A&\equiv 1+\cot^2A\\
\sin 2A &\equiv 2\sin A \cos A \\
\cos 2A &\equiv \cos^2A-\sin^2A \\
\tan 2A &\equiv \frac{2\tan A}{1-\tan^2A}
\end{align*}

\subsection{Mensuration}

Circumference and Area of circle, radius \(r\) and diameter \(d\)
\begin{align*} 
C&=2\pi r = \pi d \\
A&=\pi r^2 
\end{align*}
Pythagoras’ Theorem: In any right-angled triangle where \(a\), 
\(b\) and \(c\) are the lengths of the sides and \(c\) is the hypotenuse:
\[c^2=a^2+b^2\]

\noindent
Area of a trapezium where where \(a\) and \(b\) are the lengths of the parallel
sides and \(h\) is their perpendicular separation:
\[\text{Area}=\frac{(a+b)h}{2}\]
\par

\[
    \text{Volume of a prism} = \text{area of cross section} \times
    \text{length}
\]

\noindent
For a circle of radius \(r\), where an angle at the centre of \(\theta\) radians
subtends an arc of length \(l\) and encloses an associated sector of area \(a\):
\begin{align*}
l&=r\theta\\
a&=\frac{r^2\theta}{2}
\end{align*}
\subsection{Complex Numbers}
For two complex numbers \(z_1=r_1e^{i\theta_1}\) and \(z_2=r_2e^{i\theta_2}\)
\begin{align*}
z_1z_2 &= r_1r_2e^{i(\theta_1+\theta_2)}\\
\frac{z_1}{z_2}&= \frac{r_1}{r_2} e^{i(\theta_1-\theta_2)}
\end{align*}
\subsubsection{Loci in the Argand diagram}
\(|z-a| = r \) is a circle radius \(r\)  centred at \(a\)
\par
\(\arg(z-a) = \theta \) is a half line drawn from \(a\) at angle \(\theta\) to a
line parallel to the positive real axis
\subsubsection{Exponential Form}
\[          
e^{i\theta}=\cos\theta+i\sin\theta
\]
\subsection{Matrices}
For a $2$ by $2$ matrix
$\begin{psmallmatrix}a & b\\ c & d\end{psmallmatrix}$
the determinant
$ \Delta =  \begin{vsmallmatrix}a & b\\ c & d\end{vsmallmatrix} = ad-bc$. 
The inverse is
$
\frac{1}{\Delta} \begin{psmallmatrix}d & -b\\ -c & a\end{psmallmatrix}
$
\par

The transformation represented by matrix $\mathbf{AB}$ is the transformation
represented by matrix $\mathbf{B}$ followed by the transformation represented by
matrix $\mathbf{A}$.
\par

For matrices $\mathbf{A}$, $\mathbf{B}$: $(\mathbf{AB})^{-1}=\mathbf{A}^{-1}\mathbf{B}^{-1}$
\subsection{Algebra}
\[
    \sum_{r=1}^nr=\frac{n(n+1)}{2}
\]
For $ax^2+bx+c=0$ with roots $\alpha$ and $\beta$:
\begin{align*}
\alpha+\beta &= \frac{-b}{a}\\
\alpha\beta&=\frac{c}{a}
\end{align*}
For $ax^3+bx^2+cx+d=0$ with roots $\alpha$, $\beta$ and $\gamma$:
\begin{align*}
\sum\alpha &= \frac{-b}{a}\\
\sum\alpha\beta&=\frac{c}{a}\\
\alpha\beta\gamma&=\frac{-d}{a}
\end{align*}

\subsection{Hyperbolic Functions}
\begin{align*}
\cosh x &\equiv \frac{e^x+e^{-x}}{2}\\
\sinh x &\equiv \frac{e^x-e^{-x}}{2}\\
\tanh x &\equiv \frac{\sinh x}{\cosh x }
\end{align*}

\subsection{Calculus and Differential Equations}
\subsubsection{Differentiation}
\begin{align*}
    &\frac{d}{dx} x^n \equiv nx^{n-1}\\
    &\frac{d}{dx} \sin kx   \equiv k\cos kx\\
    &\frac{d}{dx} \cos kx   \equiv -k\sin kx\\
    &\frac{d}{dx} \sinh kx  \equiv k\cosh kx\\
    &\frac{d}{dx} \cosh kx  \equiv k\sinh kx \\
    &\frac{d}{dx} e^{kx}    \equiv ke^{kx}\\
    &\frac{d}{dx} \ln x     \equiv \frac{1}{x}\\
    &\frac{d}{dx} f(x)+g(x) \equiv f'(x)+g'(x)\\
    &\frac{d}{dx} f(x)g(x)  \equiv f'(x)g(x) + f(x)g'(x) \\
    &\frac{d}{dx} f(g(x))   \equiv f'(g(x))g'(x)
\end{align*}
\subsubsection{Integration}
\begin{align*}
    &\int x^n           \,dx \equiv \frac{x^{n+1}}{n+1} + c \ \text{where}\  n\neq -1 \\
    &\int \cos kx       \,dx \equiv \frac{\sin kx}{k} + c\\
    &\int \sin kx       \,dx \equiv \frac{-\cos kx}{k} + c\\
    &\int \cosh kx      \,dx \equiv \frac{\sinh kx}{k} + c\\
    &\int \sinh kx      \,dx \equiv \frac{\cosh kx}{k} + c\\
    &\int e^{kx}        \,dx \equiv \frac{e^{kx}}{k} + c\\
    &\int \frac{1}{x}   \,dx \equiv \ln|x|+c \,\text{where}\,x\neq 0\\
    &\int f'(x)+g'(x)   \,dx \equiv f(x)+g(x)+c\\
    &\int f'(g(x))g'(x) \,dx \equiv f(g(x))+c
\end{align*}

\subsubsection{Area under a curve}
\[
    \int_a^b y\,dx \ \text{where} \ y\geq 0
\]

\subsubsection{Volumes of revolution about the x and y axes}
\begin{align*}
    V_x &= \pi\int_a^b y^2 \,dx\\
    V_y &= \pi\int_c^d x^2 \, dy
\end{align*}

\subsubsection{Simple Harmonic Motion}
\[\ddot{x}=-\omega^2x\]

%TODO: Vectors

%TODO: Non-pure core

\chapter{Number Sets}
\section{Integers}
\begin{itemize}
    \item $\natplus$: Positive integers without $0$. $\{\,1,\, 2,\, 3\,\dots\,\}$
    \item $\natzero$: Positive integers with $0$. $\{\,0,\,1,\, 2,\, 3\,\dots\,\}$
    \item $\ints$: Integers. $\{\, \dots \,-3,\,-2,\,-1,\,0,\,1,\,2,\,3\,\dots\,\}$
\end{itemize}
\section{Unused Integers}
I will try to avoid these as they are ambiguous.
\begin{itemize}
    \item $\mathbb{N}$: Natural numbers. May or may not include $0$. I will use 
            $\natplus$ or $natzero$ to be explicit.
\end{itemize}
\section{Non-Integers}
\begin{itemize}
    \item $\rationals$: Numbers that can be expressed as a fraction: 
        $\rationals=\{\, \frac{p}{q} \,\mid\, p\in\ints,\, q\in\natplus \,\}$
    \item $\reals$: Real numbers. The definition gets axiomatic fast, so I will just
                    say that they are just $1$ number (not a matrix/vector) and are
                    not complex / imaginary.
    \item $\complexs$: A number with a real part and a complex part. 
            $\complexs = \{\, a+bi \,\mid\, a\in\reals\,b\in\reals\,\}$
                       
\end{itemize}

% \begin{tikzpicture}[fill=gray]
%     % left hand
%     \scope
%     \clip (-2,-2) rectangle (2,2)
%           (1,0) circle (1);
%     \fill (0,0) circle (1);
%     \endscope
%     % right hand
%     \scope
%     \clip (-2,-2) rectangle (2,2)
%           (0,0) circle (1);
%     \fill (1,0) circle (1);
%     \endscope
%     % outline
%     \draw (0,0) circle (1) (0,1)  node [text=black,above] {$A$}
%           (1,0) circle (1) (1,1)  node [text=black,above] {$B$}
%           (-2,-2) rectangle (3,2) node [text=black,above] {$H$};
% \end{tikzpicture}