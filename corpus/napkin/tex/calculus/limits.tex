\chapter{Limits and series}
\label{ch:calc_limits}
Now that we have developed the theory
of metric (and topological) spaces well,
we give a three-chapter sequence which
briskly covers the theory of single-variable calculus.

Much of the work has secretly already been done,
For example, if $x_n$ and $y_n$ are real sequences
with $\lim_n x_n = x$ and $\lim_n y_n = y$,
then in fact $\lim_n (x_n + y_n) = x+y$
or $\lim_n (x_n y_n) = xy$,
because we showed in \Cref{prop:arithmetic_continuous}
that arithmetic was continuous.
We will also see that completeness plays a crucial role.

\section{Completeness and inf/sup}
\prototype{$\sup [0,1] = \sup (0,1) =  1$.}
As $\RR$ is a metric space,
we may discuss continuity and convergence.
There are two important facts about $\RR$
which will make most of the following sections tick.

The first fact you have already seen before:
\begin{theorem}
	[$\RR$ is complete]
	\label{thm:complete_R}
	As a metric space, $\RR$ is complete:
	sequences converge if and only if they are Cauchy.
\end{theorem}

The second one we have not seen before --- it is the
existence of $\inf$ and $\sup$.
Your intuition should be:
\begin{moral}
	$\sup$ is $\max$ adjusted slightly for infinite sets.
	(And $\inf$ is adjusted $\min$.)
\end{moral}
Why the ``adjustment''?
\begin{example}
	[Why is max not good enough?]
	Let's say we have the open interval $S = (0,1)$.
	The elements can get arbitrarily close to $1$,
	so we would like to think ``$1$ is the max of $S$'';
	except the issue is that $1 \notin S$.
	In general, infinite sets don't necessarily \emph{have}
	a maximum, and we have to talk about bounds instead.

	So we will define $\sup S$ in such a way that $\sup S = 1$.
	The definition is that ``$1$ is the smallest number
	which is at least every element of $S$''.
\end{example}
To write it out:
\begin{definition}
	If $S$ is a set of real numbers:
	\begin{itemize}
		\ii An \emph{upper bound} for $S$ is a real number $M$
		such that $x \le M$ for all $x \in S$.
		If one exists, we say $S$ is \vocab{bounded above};
		\ii A \emph{lower bound} for $S$ is a real number $m$
		such that $m \le x$ for all $x \in S$.
		If one exists, we say $S$ is \vocab{bounded below}.
		\ii If both upper and lower bounds exist,
		we say $S$ is \vocab{bounded}.
	\end{itemize}
\end{definition}
\begin{theorem}
	[$\RR$ has inf's and sup's]
	\label{thm:inf_sup}
	Let $S$ be a nonempty set of real numbers.
	\begin{itemize}
		\ii If $S$ is bounded above
		then it has a \emph{least} upper bound,
		which we denote by $\sup S$
		and refer to as the \vocab{supremum} of $S$.
		\ii If $S$ is bounded below
		then it has a \emph{greatest} lower bound,
		which we denote by $\inf S$
		and refer to as the \vocab{infimum} of $S$.
	\end{itemize}
\end{theorem}

\begin{definition}
	For convenience, if $S$ has not bounded above, we write $\sup S = +\infty$.
	Similarly, if $S$ has not bounded below, we write $\inf S = -\infty$.
\end{definition}

\begin{example}
	[Supremums]
	Since the examples for infimums are basically the same,
	we stick with supremums for now.
	\begin{enumerate}[(a)]
		\ii If $S = \left\{ 1, 2, 3, \dots \right\}$
		then $S$ is not bounded above, so we have $\sup S = +\infty$.
		\ii If $S = \left\{ \dots, -2, -1 \right\}$
		denotes the set of negative integers, then $\sup S = -1$.
		\ii Let $S = [0,1]$ be a closed interval.
		Then $\sup S = 1$.
		\ii Let $S = (0,1)$ be an open interval.
		Then $\sup S = 1$ as well, even though $1$ itself
		is not an element of $S$.
		\ii Let $S = \QQ \cap (0,1)$ denote the set of rational
		numbers between $0$ and $1$.
		Then $\sup S = 1$ still.
		\ii If $S$ is a finite nonempty set,
		then $\sup S = \max S$.
	\end{enumerate}
\end{example}

\begin{definition}
	[Porting definitions to sequences]
	If $a_1$, \dots is a sequence we will often write
	\begin{align*}
		\sup_n a_n &\defeq \sup \left\{ a_n \mid n \in \NN \right\} \\
		\inf_n a_n &\defeq \inf \left\{ a_n \mid n \in \NN \right\}
	\end{align*}
	for the supremum and infimum of the set of elements of the sequence.
	We also use the words ``bounded above/below'' for sequences
	in the same way.
\end{definition}
\begin{example}
	[Infimum of a sequence]
	The sequence $a_n = \frac 1n$ has infimum $\inf a_n = 0$.
\end{example}

\section{Proofs of the two key completeness properties of $\RR$}
Careful readers will note that we have not actually
proven either \Cref{thm:inf_sup} or \Cref{thm:complete_R}.
We will do so here.

First, we show that the ability
to take infimums and supremums lets you prove completeness of $\RR$.
\begin{proof}
	[Proof that \Cref{thm:inf_sup} implies \Cref{thm:complete_R}]
	Let $a_1$, $a_2$, \dots be a Cauchy sequence.
	By discarding finitely many leading terms,
	we may as well assume that $|a_i - a_j| \le 100$ for all $i$ and $j$.
	In particular, the sequence is now bounded;
	it lies between $[a_1-100, a_1+100]$ for example.

	We want to show this sequence converges,
	so we have to first describe what the limit is.
	We know that to do this we are really going
	to have to use the fact that we live in $\RR$.
	(For example we know in $\QQ$ the limit of
	$1$, $1.4$, $1.41$, $1.414$, \dots is nonexistent.)

	We propose the following: let
	\[ S = \left\{ x \in \RR \mid a_n \ge x \text {
		for infinitely many $n$ } \right\}.  \]
	We claim that the sequence converges to $M = \sup S$.
	\begin{exercise}
		Show that this supremum makes sense by proving
		that $a_1 - 100 \in S$ (so $S$ is nonempty)
		while all elements of $S$ are at most $a_1 + 100$
		(so $S$ is bounded above).
		Thus we are allowed to actually take the supremum.
	\end{exercise}

	You can think of this set $S$ with the following picture.
	We have a Cauchy sequence drawn in the real line which we think converges,
	which we can visualize as a bunch of dots on the real line,
	with some order on them.
	We wish to cut the line with a knife such that
	only finitely many dots are do to the left of the knife.
	(For example, placing the knife all the way to the left always works.)
	The set $S$ represents the places where we could put the knife,
	and $M$ is ``as far right'' as we could go.
	Because of the way supremums work,
	$M$ might not \emph{itself} be a valid knife location,
	but certainly anything to its left is.
	\begin{center}
	\begin{asy}
		size(10cm);
		draw( (-8,0)--(8,0), Arrows );
		label("$\mathbb R$", (8,0), dir(-90));
		draw( (3.1,1)--(3.1,-1), red );
		draw( (2.7,1.5)--(2.7,-1.5), deepgreen );
		label("$M$", (3.1,-1), dir(-45), red);
		label("$M-\frac12\varepsilon$", (2.7,1.5), dir(90), deepgreen);
		dot("$a_1$", (-5, 0), dir(-90), blue);
		dot("$a_2$", (-1, 0), dir(-90), blue);
		dot("$a_3$", ( 6, 0), dir(-90), blue);
		dot("$a_4$", ( 0, 0), dir(-90), blue);
		dot("$a_5$", (4.8, 0), dir(-90), blue);
		dot("$a_6$", (4.1, 0), dir(-90), blue);
		dot("$a_8$", (3.5, 0), dir(-90), blue);
		dot("$a_7$", (1.5, 0), dir(-90), blue);
		dot((3.3, 0), blue);
		dot((3.2, 0), blue);
		dot((3.05, 0), blue);
	\end{asy}
	\end{center}
	Let $\eps > 0$ be given;
	we want to show eventually all terms are within $\eps$ of $M$.
	Because the sequence is Cauchy,
	there is an $N$ such that
	eventually $\left\lvert a_m - a_n \right\rvert < \half\eps$
	for $m \ge n \ge N$.

	Now suppose we fix $n$ and vary $m$.
	By the definition of $M$,
	it should be possible to pick the index $m$
	such that $a_m \ge M - \half \eps$
	(there are infinitely many to choose from
	since $M - \half\eps$ is a valid knife location,
	and we only need $m \ge n$).
	In that case we have
	\[ \left\lvert a_n - M \right\rvert
		\le \left\lvert a_n - a_m \right\rvert
		+ \left\lvert a_m - M \right\rvert <
		\half\eps + \half\eps = \eps \]
	by the triangle inequality.
	This completes the proof.
\end{proof}

Therefore it is enough to prove the latter \Cref{thm:inf_sup}.
To do this though, we would need to actually give a rigorous definition of
the real numbers $\RR$, since we have not done so yet!

One approach that makes this easy to use the so-called
\vocab{Dedekind cut} construction.
Suppose we take the rational numbers $\QQ$.
Then one \emph{defines} a real number to be a ``cut''
$A \mid B$ of the set of rational numbers:
a pair of subsets of $\QQ$ such that
\begin{itemize}
	\ii $\QQ = A \sqcup B$ is a disjoint union;
	\ii $A$ and $B$ are nonempty;
	\ii we have $a < b$ for every $a \in A$ and $b \in B$, and
	\ii $A$ has no largest element (i.e.\ $\sup A \notin A$).
\end{itemize}
This can again be visualized by taking what you think of
as the real line, and slicing at some real number.
The subset $\QQ \subset \RR$ gets cut into two halves $A$ and $B$.
If the knife happens to land exactly at a rational number,
by convention we consider that number to be in the right half
(which explains the last fourth condition that $\sup A \notin A$).

With this definition \Cref{thm:inf_sup} is easy:
to take the supremum of a set of real numbers,
we take the union of all the left halves.
The hard part is then figuring out how to define $+$, $-$, $\times$, $\div$
and so on with this rather awkward construction.
If you want to read more about this construction in detail,
my favorite reference is \cite{ref:pugh}$,$
in which all of this is done carefully in Chapter 1.

\section{Monotonic sequences}
Here is a great exercise.
\begin{exercise}
	[Mandatory]
	\label{exer:inf_exists}
	Prove that if $a_1 \ge a_2 \ge \dots \ge 0$
	then the limit
	\[ \lim_{n \to \infty} a_n \]
	exists.
	Hint: the idea in the proof of the previous section helps;
	you can also try to use completeness of $\RR$.
	Second hint: if you are really stuck,
	wait until after \Cref{thm:nonneg_bounded},
	at which point you can use essentially copy its proof.
\end{exercise}

The proof here readily adapts by shifting.
\begin{definition}
	A sequence $a_n$ is \vocab{monotonic}
	if either $a_1 \ge a_2 \ge \dots$
	or $a_1 \le a_2 \le \dots$.
\end{definition}

\begin{theorem}
	[Monotonic bounded sequences converge]
	Let $a_1$, $a_2$, \dots be a monotonic bounded sequence.
	Then $\lim_{n \to \infty} a_n$ exists.
	\label{thm:monotonic_bounded}
\end{theorem}

\begin{example}
	[Silly example of monotonicity]
	Consider the sequence defined by
	\begin{align*}
		a_1 &= 1.2 \\
		a_2 &= 1.24 \\
		a_3 &= 1.248 \\
		a_4 &= 1.24816 \\
		a_5 &= 1.2481632 \\
		&\vdotswithin=
	\end{align*}
	and so on, where in general we stuck
	on the decimal representation of the next power of $2$.
	This will converge to \emph{some} real number,
	although of course this number
	is quite unnatural and there is probably no good description for it.
\end{example}
In general, ``infinite decimals''
can now be defined as the limit of the truncated finite ones.

\begin{example}
	[$0.9999\dots = 1$]
	In particular, I can finally make precise the notion
	you argued about in elementary school that
	\[ 0.9999\dots = 1. \]
	We simply \emph{define} a repeating decimal
	to be the limit of the sequence $0.9$, $0.99$, $0.999\dots$.
	And it is obvious that the limit of this sequence is $1$.
\end{example}

Some of you might be a little surprised since
it seems like we really should have
$0.9999 = 9 \cdot 10^{-1} + 9 \cdot 10^{-2} + \dots$ ---
the limit of ``partial sums''.
Don't worry, we're about to define those in just a moment.

\medskip

Here is one other great use of monotonic sequences.
\begin{definition}
	Let $a_1$, $a_2$, \dots be a sequence
	(not necessarily monotonic) which is bounded below.
	We define
	\[
		\limsup_{n \to \infty} a_n
		\defeq \lim_{N \to \infty} \sup_{n \ge N} a_n
		= \lim_{N \to \infty} \sup \left\{ a_N, a_{N+1}, \dots \right\}.
	\]
	This is called the \vocab{limit supremum} of $(a_n)$.
	We set $\limsup_{n \to \infty} a_n$ to be $+\infty$
	if $a_n$ is not bounded above.

	If $a_n$ is bounded above,
	the \vocab{limit infimum} $\liminf_{n \to \infty} a_n$
	is defined similarly.
	In particular, $\liminf_{n \to \infty} a_n = -\infty$
	if $a_n$ is not bounded below.
\end{definition}
\begin{exercise}
	Show that these definitions make sense,
	by checking that the supremums are non-increasing,
	and bounded below.
\end{exercise}
We can think of $\limsup_n a_n$ as
``supremum, but allowing finitely many terms to be discarded''.


\section{Infinite series}
\prototype{$\sum_{k \ge 1}^\infty \frac{1}{k(k+1)}
= \lim_{n \to \infty} \left( 1 - \frac{1}{n+1} \right) = 1$.}

We will actually begin by working with infinite series,
since in the previous chapters we defined limits of sequences,
and so this is actually the next closest thing to work
with.\footnote{Conceptually: discrete things are easier
	to be rigorous about than continuous things,
	so series are actually ``easier'' than derivatives!
	I suspect the reason that most schools teach series last in calculus
	is that most calculus courses do not have proofs.}

This will give you a rigorous way to think about
statements like
\[ \sum_{n = 1}^{\infty} \frac{1}{n^2} = \frac{\pi^2}{6} \]
and help answer questions like
``how can you add rational numbers and get an irrational one?''.

\begin{definition}
	Consider a sequence $a_1$, \dots{} of real numbers.
	The series $\sum_k a_k$ \vocab{converges} to a limit $L$ if
	the sequence of ``partial sums''
	\begin{align*}
		s_1 &= a_1 \\
		s_2 &= a_1 + a_2 \\
		s_3 &= a_1 + a_2 + a_3 \\
		&\vdotswithin= \\
		s_n &= a_1 + \dots + a_n
	\end{align*}
	converges to the limit $L$.
	Otherwise it \vocab{diverges}.
\end{definition}
\begin{abuse}
	[Writing divergence as $+\infty$]
	It is customary, if all the $a_k$ are nonnegative,
	to write $\sum_k a_k = \infty$ to denote that the series diverges.
\end{abuse}

You will notice that by using the definition of sequences,
we have masterfully sidestepped the issue of
``adding infinitely many numbers''
which would otherwise cause all sorts of problems.
\begin{moral}
	An ``infinite sum'' is actually the \emph{limit} of its partial sums.
	There is no infinite addition involved.
\end{moral}

That's why it's for example okay to have
$\sum_{n \ge 1} \frac{1}{n^2} = \frac{\pi^2}{6}$ be irrational;
we have already seen many times that sequences of rational
numbers can converge to irrational numbers.
It also means we can gladly ignore all the
irritating posts by middle schoolers about
$1+2+3+\dots = -\frac{1}{12}$;
the partial sums explode to $+\infty$, end of story,
and if you want to assign a value to that sum
it had better be a definition.

\begin{example}
	[The classical telescoping series]
	We can now prove the classic telescoping series
	\[ \sum_{k = 1}^\infty \frac{1}{k(k+1)} \]
	in a way that doesn't just hand-wave the ending.
	Note that the $k$th partial sum is
	\begin{align*}
		\sum_{k=1}^n \frac{1}{k(k+1)}
		&= \frac{1}{1 \cdot 2}
			+ \frac{1}{2 \cdot 3}
			+ \dots + \frac{1}{n(n+1)} \\
		&= \left( \frac11 - \frac12 \right)
			+ \dots + \left( \frac 1n - \frac{1}{n+1} \right) \\
		&= 1 - \frac{1}{n+1}.
	\end{align*}
	The limit of this partial sum as $n \to \infty$ is $1$.
\end{example}

\begin{example}
	[Harmonic series diverges]
	We can also make sense of the statement
	that $\sum_{k=1}^\infty \frac 1k = \infty$
	(i.e.\ it diverges).
	We may bound the $2^n$th partial sums from below:
	\begin{align*}
		\sum_{k=1}^{2^n} \frac1k &= \frac11 + \frac12 + \dots + \frac1{2^n} \\
		&\ge \frac11 + \frac12 + \left( \frac14+\frac14 \right)
		+ \left( \frac18+\frac18+\frac18+\frac18 \right) \\
		&+ \dots  +
		\underbrace{\left( \frac{1}{2^n} + \dots + \frac{1}{2^n} \right)}_{2^{n-1} \text{ terms}} \\
		&= 1 + \half + \half + \dots + \half = 1 + \frac{n-1}{2}.
	\end{align*}
	A sequence satisfying $s_{2^n} \ge 1 + \half(n-1)$
	will never converge to a finite number!
\end{example}

I had better also mention that for nonnegative sums,
convergence is just the same as having ``finite sum''
in the following sense.

\begin{proposition}
	[Partial sums of nonnegatives bounded implies convergent]
	\label{thm:nonneg_bounded}
	Let $\sum_k a_k$ be a series of \emph{nonnegative} real numbers.
	Then $\sum_k a_k$ converges to some limit
	if and only if there is a constant $M$ such that
	\[ a_1 + \dots + a_n < M \]
	for every positive integer $n$.
\end{proposition}
\begin{proof}
	This is actually just \Cref{thm:monotonic_bounded} in disguise,
	but since we left the proof as an exercise back then,
	we'll write it out this time.

	Obviously if no such $M$ exists then convergence will not happen,
	since this means the sequence $s_n$ of partial sums is unbounded.

	Conversely, if such $M$ exists then we have
	$s_1 \le s_2 \le \dots < M$.
	Then we contend the sequence $s_n$ converges to
	$L \defeq \sup_n s_n < \infty$.
	(If you read the proof that completeness implies Cauchy,
	the picture is nearly the same here, but simpler.)
	\begin{center}
	\begin{asy}
		size(10cm);
		draw( (-8,0)--(8,0), Arrows );
		label("$\mathbb R$", (8,0), dir(-90));
		draw( (3.1,1)--(3.1,-1), red );
		draw( (2.4,1.5)--(2.4,-1.5), deepgreen );
		label("$L$", (3.1,-1), dir(-45), red);
		label("$L-\varepsilon$", (2.4,1.5), dir(90), deepgreen);
		dot("$s_1$", (-5, 0), dir(-90), blue);
		dot("$s_2$", (-4, 0), dir(-90), blue);
		dot("$s_3$", (-1.7, 0), dir(-90), blue);
		dot("$s_4$", (-0.2, 0), dir(-90), blue);
		dot("$s_5$", (0.3, 0), dir(-90), blue);
		dot("$s_6$", (1.3, 0), dir(-90), blue);
		dot("$s_7$", (2.2, 0), dir(-90), blue);
		dot("$s_8$", (2.7, 0), dir(-90), blue);
		dot((2.8, 0), blue);
		dot((2.88, 0), blue);
		dot((2.97, 0), blue);
	\end{asy}
	\end{center}

	Indeed, this means for any $\eps$ there
	are infinitely many terms of the sequence exceeding $L-\eps$;
	but since the sequence is monotonic, once $s_n \ge L-\eps$
	then $s_{n'} \ge L-\eps$ for all $n' \ge n$.
	This implies convergence.
\end{proof}

\begin{abuse}
	[Writing $\sum < \infty$]
	For this reason, if $a_k$ are nonnegative real numbers,
	it is customary to write \[ \sum_k a_k < \infty \]
	as a shorthand for ``$\sum_k a_k$ converges to a finite limit'',
	(or perhaps shorthand for ``$\sum_k a_k$ is bounded'' --- as
	we have just proved these are equivalent).
	We will use this notation too.
\end{abuse}

\section{Series addition is not commutative: a horror story}
One unfortunate property of the above definition
is that it actually depends on the order of the elements.
In fact, it turns out that there is an explicit way
to describe when rearrangement is okay.

\begin{definition}
	A series $\sum_k a_k$ of real numbers
	is said to \vocab{converge absolutely} if
	\[ \sum_k \left\lvert a_k \right\rvert < \infty \]
	i.e.\ the series of absolute values converges to some limit.
	If the series converges, but not absolutely,
	we say it \vocab{converges conditionally}.
\end{definition}

\begin{proposition}
	[Absolute convergence $\implies$ convergence]
	If a series $\sum_k a_k$ of real numbers
	converges absolutely, then it converges in the usual sense.
\end{proposition}
\begin{exercise}
	[Great exercise]
	Prove this by using the Cauchy criteria:
	show that if the partial sums of $\sum_k |a_k|$ are Cauchy,
	then so are the partial sums of $\sum_k a_k$.
\end{exercise}

Then, rearrangement works great.
\begin{theorem}
	[Permutation of terms okay for absolute convergence]
	Consider a series $\sum_k a_k$ which is absolutely convergent
	and has limit $L$.
	Then any permutation of the terms will also converge to $L$.
\end{theorem}
\begin{proof}
	Suppose $\sum_k a_k$ converges to $L$,
	and $b_n$ is a rearrangement.
	Let $\eps > 0$.
	We will show that the partial sums of $b_n$
	are eventually within $\eps$ of $L$.

	The hypothesis means that there is a large $N$ in terms of $\eps$
	such that
	\[ \left\lvert \sum_{k=1}^N a_k - L \right\rvert < \half\eps
		\quad\text{and}\quad
		\sum_{k = N+1}^{n} \left\lvert a_k \right\rvert < \half\eps
	\]
	for every $n \ge N$ (the former from vanilla convergence of $a_k$
	and the latter from the fact that $a_k$ converges absolutely,
	hence its partial sums are Cauchy).

	Now suppose $M$ is large enough that $a_1$, \dots, $a_N$
	are contained within the terms $\{b_1, \dots, b_M\}$.
	Then
	\begin{align*}
		b_1 + \dots + b_M
		&= (a_1 + \dots + a_N) \\
		&+ \underbrace{a_{i_1} + a_{i_2} + \dots + a_{i_{M-N}}}%
		_{\text{$M-N$ terms with indices $\ge N$}}
	\end{align*}
	The terms in the first line sum up to within $\half\eps$ of $L$,
	and the terms in the second line have sum at most $\half\eps$
	in absolute value, so the total $b_1 + \dots + b_M$
	is within $\half\eps + \half\eps = \eps$ of $L$.
\end{proof}
In particular, when you have nonnegative terms, the world is great:
\begin{moral}
	Nonnegative series can be rearranged at will.
\end{moral}
And the good news is that actually,
in practice, most of your sums will be nonnegative.

The converse is not true,
and in fact, it is almost the worst possible converse you can imagine.
\begin{theorem}
	[Permutation of terms meaningless for conditional convergence]
	Consider a series $\sum_k a_k$ which converges \emph{conditionally}
	to some real number.
	Then, there exists a permutation of the series
	which converges conditionally to $1337$.

	(Or any constant.  You can also get it to diverge, too.)
\end{theorem}
So, permutation is as bad as possible for conditionally convergent
series, and hence don't even bother to try.

\section{Limits of functions at points}
\prototype{$\lim_{x \to \infty} 1/x = 0$.}
We had also better define the notion
of a limit of a real function,
which (surprisingly) we haven't actually defined yet.
The definition will look like what we have seen before with continuity.

\begin{definition}
	Let $f \colon \RR \to \RR$ be a function\footnote{Or
		$f \colon (a,b) \to \RR$, or variants.
		We just need $f$ to be defined on an open neighborhood of $p$.
	} and let $p \in \RR$ be a point in the domain.
	Suppose there exists a real number $L$ such that:
	\begin{quote}
		For every $\eps > 0$, there exists $\delta > 0$
		such that if $\left\lvert x - p \right\rvert < \delta$
		and $x \neq p$ then $\left\lvert f(x) - L \right\rvert < \eps$.
	\end{quote}
	Then we say $L$ is the \vocab{limit} of $f$ as $x \to p$, and write
	\[ \lim_{x \to p} f(x) = L. \]
\end{definition}
There is an important point here: in this definition
we \emph{deliberately} require that $x \ne p$.
\begin{moral}
	The value $\lim_{x \to p} f(x)$ does not depend on $f(p)$,
	and accordingly we often do not even bother to define $f(p)$.
\end{moral}
\begin{example}
	[Function with a hole]
	Define the function $f \colon \RR \to \RR$ by
	\[ f(x) = \begin{cases}
			3x & \text{if } x \ne 0 \\
			2019 & \text{otherwise}.
		\end{cases} \]
	Then $\lim_{x \to 0} f(x) = 0$.
	The value $f(0) = 2019$ does not affect the limit.
	Obviously, because $f(0)$ was made up to be some artificial
	value that did not agree with the limit,
	this function is discontinuous at $x = 0$.
\end{example}

\begin{ques}
	[Mandatory]
	Show that a function $f$ is continuous at $p$
	if and only if $\lim_{x \to p} f(x)$ exists and equals $f(p)$.
\end{ques}

\begin{example}
	[Less trivial example: a rational piecewise function]
	\label{ex:rational_piecewise}
	Define the function $f \colon \RR \to \RR$ as follows:
	\[ f(x) = \begin{cases}
			1 & \text{if } x = 0 \\
			\frac1q & \text{if } x = \frac pq \text{ where } q > 0
				\text{ and } \gcd(p,q) = 1 \\
			0 & \text{if } x \notin \QQ.
		\end{cases} \]
	For example, $f(\pi) = 0$, $f(2/3) = \frac13$, $f(0.17) = \frac{1}{100}$.
	Then \[ \lim_{x \to 0} f(x) = 0. \]
	For example, if $|x| < 1/100$ and $x \neq 0$
	then $f(x)$ is either zero (for $x$ irrational)
	or else is at most $\frac{1}{101}$ (if $x$ is rational).

	As $f(0) = 1$, this function is also discontinuous at $x = 0$.
	However, if we change the definition so that $f(0) = 0$ instead,
	then $f$ becomes continuous at $0$.
\end{example}

\begin{example}
	[Famous example]
	Let $f(x) = \frac{\sin x}{x}$, $f \colon \RR \to \RR$,
	where $f(0)$ is assigned any value.
	Then
	\[ \lim_{x \to 0} f(x) = 1. \]
\end{example}
We will not prove this here,
since I don't want to get into trig yet.
In general, I will basically only use trig functions
for examples and not for any theory,
so most properties of the trig functions will just be quoted.
\begin{abuse}
	[The usual notation]
	\label{abuse:limit}
	From now on, the above example
	will usually be abbreviated to just
	\[ \lim_{x \to 0} \frac{\sin x}{x} = 1. \]
	The reason there is a slight abuse here
	is that I'm supposed to feed a function $f$ into the limit,
	and instead I've written down an expression
	which is defined everywhere --- except at $x=0$.
	But that $f(0)$ value doesn't change anything.
	So the above means: ``the limit of the function described
	by $f(x) = \frac{\sin x}{x}$,
	except $f(0)$ can be whatever it wants because it doesn't matter''.
\end{abuse}

\begin{remark}
	[For metric spaces]
	You might be surprised that I didn't define
	the notion of $\lim_{x \to p} f(x)$ earlier
	for $f \colon M \to N$ a function on metric spaces.
	We can actually do so as above, but there is one nuance:
	what if our metric space $M$ is discrete,
	so $p$ has no points nearby it?
	(Or even more simply, what if $M$ is a one-point space?)
	We then cannot define $\lim_{x \to p} f(x)$ at all.

	Thus if $f \colon M \to N$ and we want to define
	$\lim_{x \to p} f(x)$, we have the requirement that $p$
	should have a point within $\eps$ of it, for any $\eps > 0$.
	In other words, $p$ should not be an isolated point.
\end{remark}

As usual, there are no surprises with arithmetic,
we have $\lim_{x \to p} (f(x) \pm g(x))
	= \lim_{x \to p} f(x) \pm \lim_{x \to p} g(x)$,
and so on and so forth.
We have effectively done this proof before
so we won't repeat it again.

\section{Limits of functions at infinity}
Annoyingly, we actually have to make this
definition separately,
even though it will not feel any different
from earlier examples.
\begin{definition}
	Let $f \colon \RR \to \RR$.
	Suppose there exists a real number $L$ such that:
	\begin{quote}
		For every $\eps > 0$, there exists a constant $M$
		such that if $x > M$, then $\left\lvert f(x)-L \right\rvert < \eps$.
	\end{quote}
	Then we say $L$ is the \vocab{limit} of $f$ as
	$x$ approaches $\infty$ and write
	\[ \lim_{x \to \infty} f(x) = L. \]
	The limit $\lim_{x \to -\infty} f(x)$ is defined similarly,
	with $x > M$ replaced by $x < M$.
\end{definition}
Fortunately, as $\infty$ is not an element of $\RR$,
we don't have to do the same antics about $f(\infty)$
like we had to do with ``$f(p)$ set arbitrarily''.
So these examples can be more easily written down.
\begin{example}
	[Limit at infinity]
	The usual:
	\[ \lim_{x \to \infty} \frac 1x = 0. \]
	I'll even write out the proof:
	for any $\eps > 0$, if $x > 1/\eps$
	then $\left\lvert \frac 1x - 0 \right\rvert < \eps$.
\end{example}

There are no surprises with arithmetic:
we have $\lim_{x \to \infty} (f(x) \pm g(x))
	= \lim_{x \to \infty} f(x) \pm \lim_{x \to p} g(x)$,
and so on and so forth.
This is about the fourth time
I've mentioned this, so I will not say more.

\section{\problemhead}

\begin{problem}
	Define the sequence
	\[ a_n = (-1)^n + \frac{n^3}{2^n} \]
	for every positive integer $n$.
	Compute the limit infimum and the limit supremum.
\end{problem}

\begin{problem}
	For which bounded sequences $a_n$
	does $\liminf_n a_n = \limsup_n a_n$?
	\begin{hint}
		Iff the sequence is convergent!
	\end{hint}
\end{problem}

\begin{dproblem}
	[Comparison test]
	Let $\sum a_n$ and $\sum b_n$ be two series.
	Assume $\sum b_n$ is absolutely convergent,
	and $|a_n| \le |b_n|$ for all integers $n$.
	Prove that $\sum_n a_n$ is absolutely convergent.
\end{dproblem}

\begin{problem}
	[Geometric series]
	\label{prob:geometric}
	Let $-1 < r < 1$ be a real number.
	Show that the series
	\[ 1 + r + r^2 + r^3 + \dots \]
	converges absolutely and determine what it converges to.
	\begin{hint}
		The $n$th partial sum is $\frac{1}{1-r} (1-r^{n+1})$.
	\end{hint}
\end{problem}

%\begin{problem}
%	[An order-free definition for infinite nonnegative sums]
%	Let $a_1$, \dots{} be a sequence of \emph{nonnegative} real numbers.
%	Prove that
%	\[ \sum_{k=1}^\infty a_k = \sup \left\{ \sum_{k \in K} a_k
%			\mid K \subseteq \NN \right\} \]
%	where if the right-hand side is $+\infty$,
%	we mean that the left-hand side is a divergent series.
%\end{problem}

\begin{problem}
	[Alternating series test]
	Let $a_0 \ge a_1 \ge a_2 \ge a_3 \ge \dots$ be a weakly decreasing sequence
	of nonnegative real numbers,
	and assume that $\lim_{n \to \infty} a_n = 0$.
	Show that the series $\sum_n (-1)^n a_n$ is convergent
	(it need not be absolutely convergent).
	\begin{sol}
		This is an application of Cauchy convergence, since
		one can show that
		\[ \left\lvert \sum_{n=M}^N (-1)^n a_n \right\rvert \le a_{\min\{M,N\}}. \]
		Indeed, if $M$ and $N$ are even
		(for simplicity; other cases identical) then
		\begin{align*}
			a_M - a_{M+1} + a_{M+2} - \dots
			&= a_M - (a_{M+1}-a_{M+2}) - (a_{M+3}-a_{M+4}) \\
			&\quad- \dots - (a_{N-1} - a_N) \\
			&\le a_M \\
			a_M - a_{M+1} + a_{M+2} - \dots
			&= a_M - a_{M+1} + (a_{M+2} - a_{M+3}) + (a_{M+4} - a_{M+5}) \\
			&\quad+ \dots + (a_{N-2} - a_{N+1}) + a_N \\
			&\ge -a_{M+1}.
		\end{align*}
		In this way we see that the sequence of partial sums is Cauchy,
		hence converges to some limit.
	\end{sol}
\end{problem}


\begin{problem}
	[{\cite[Chapter 3, Exercise 55]{ref:pugh}}]
	\gim
	Let $(a_n)_{n \ge 1}$ and $(b_n)_{n \ge 1}$ be sequences of real numbers.
	Assume $a_1 \le a_2 \le \dots \le 1000$
	and moreover that $\sum_n b_n$ converges.
	Prove that $\sum_n a_n b_n$ converges.
	(Note that in both the hypothesis and statement,
	we do not have absolute convergence.)
	\begin{hint}
		This is a very tricky algebraic manipulation.
		Try setting $a_n = x_1 + \dots + x_n$ for $x_i \ge 0$.
	\end{hint}
	\begin{sol}
		To capture $(a_n)_n$ monotonic and bounded,
		write $a_n = x_1 + \dots + x_n$ for some $x_i$.
		Then $x_2$, \dots are all the same sign and so $\sum |x_i| = A < \infty$
		for some constant $A$.

		We now prove that the partial sums of $\sum a_n b_n$ are a Cauchy sequence.
		Consider any $\varepsilon > 0$.
		Let $K$ be such that the tails of $b_n$
		starting after $K$ have absolute value less than $\frac{\varepsilon}{A}$.
		Then for any $N > M \ge K$ we have
		\begin{align*}
			\left\lvert \sum_{k=M}^N a_k b_k \right\rvert
			&= \left\lvert \sum_{k=M}^N \sum_{j=1}^k b_k x_j \right\rvert \\
			&= \left\lvert \sum_{j=1}^N \sum_{k=\max\{j,M\}}^N b_k x_j \right\rvert \\
			&= \left\lvert \sum_{j=1}^N x_j \cdot \sum_{k=\max\{j,M\}}^N b_k\right\rvert \\
			&\le \sum_{j=1}^N |x_j| \left\lvert \sum_{k=\max\{j,M\}}^N b_k \right\rvert \\
			&< \sum_{j=1}^N |x_j| \cdot \frac{\varepsilon}{A} \\
			&< \varepsilon
		\end{align*}
		as desired.
	\end{sol}
\end{problem}

\begin{problem}
	[Putnam 2016 B1]
	\gim
	Let $x_0, x_1, x_2, \dots$ be the sequence
	such that $x_0=1$ and for $n\ge 0$,
	\[ x_{n+1} = \log(e^{x_n}-x_n) \]
	(as usual, $\log$ is the natural logarithm).
	Prove that the infinite series $x_0 + x_1 + \dots$
	converges and determine its value.
	\begin{hint}
		This is trickier than it looks.
		We have $x_n = e^{x_n} - e^{x_{n+1}}$
		but it requires some care to prove convergences.
		Helpful hint: $e^t \ge t+1$ for all real numbers $t$,
		therefore all $x_n$'s are nonnegative.
	\end{hint}
	\begin{sol}
		The answer is $e-1$.

		We begin by noting $x_{n+1} = \log(e^{x_n} - x_n) \ge \log 1 = 0$,
		owing to $e^t \ge 1+t$.
		So $x_n \ge 0$ for all $n$.

		Next notice that
		\[ x_{n+1} = \log\left( e^{x_n} - x_n \right) < \log e^{x_n} = x_n. \]
		So $(x_n)_n$ is strictly decreasing in addition to nonnegative.
		Thus it must converge to some limit $L$.

		Third, observe that
		\[ x_n = e^{x_n} - e^{x_{n+1}}
			\implies x_0 + x_1 + \dots + x_n
			= e^{x_0} - e^{x_n} = e - e^{x_n} < e. \]
		Since the partial sums are bounded by $e$,
		and $x_i \ge 0$, we conclude $L = 0$.

		Finally, the limit of the partial sums is then
		\[ \lim_{n \to \infty} e - e^{x_n} = e - e^0 = e - 1. \]
	\end{sol}
\end{problem}


\begin{problem}
	Consider again the function $f \colon \RR \to \RR$
	in \Cref{ex:rational_piecewise} defined by
	\[ f(x) = \begin{cases}
			1 & \text{if } x = 0 \\
			\frac1q & \text{if } x = \frac pq
			\text{ where } q > 0 \text{ and } \gcd(p,q) = 1 \\
			0 & \text{if } x \notin \QQ.
		\end{cases} \]
	For every real number $p$,
	compute $\lim_{x \to p} f(x)$, if it exists.
	At which points is $f$ continuous?
	\begin{hint}
		The limit always exists and equals zero.
		Consequently, $f$ is continuous exactly at irrational points.
	\end{hint}
\end{problem}

